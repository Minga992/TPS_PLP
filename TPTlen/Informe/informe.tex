\documentclass[a4paper]{article}
\usepackage[spanish]{babel}
\usepackage[utf8]{inputenc}
\usepackage{fancyhdr}
\usepackage{charter}   % tipografía
\usepackage{graphicx}
\usepackage{makeidx}

\usepackage{float}
\usepackage{amsmath, amsthm, amssymb}
\usepackage{amsfonts}
\usepackage{sectsty}
\usepackage{wrapfig}
\usepackage{listings} % necesario para el resaltado de sintaxis
\usepackage{caption}
\usepackage{placeins}
\usepackage{longtable}

\usepackage{hyperref} % agrega hipervínculos en cada entrada del índice
\hypersetup{          % (en el pdf)
    colorlinks=true,
    linktoc=all,
    citecolor=black,
    filecolor=black,
    linkcolor=black,
    urlcolor=black
}

\usepackage{color} % para snippets de código coloreados
\usepackage{fancybox}  % para el sbox de los snippets de código

\definecolor{litegrey}{gray}{0.94}

% \newenvironment{sidebar}{%
% 	\begin{Sbox}\begin{minipage}{.85\textwidth}}%
% 	{\end{minipage}\end{Sbox}%
% 		\begin{center}\setlength{\fboxsep}{6pt}%
% 		\shadowbox{\TheSbox}\end{center}}
% \newenvironment{warning}{%
% 	\begin{Sbox}\begin{minipage}{.85\textwidth}\sffamily\lite\small\RaggedRight}%
% 	{\end{minipage}\end{Sbox}%
% 		\begin{center}\setlength{\fboxsep}{6pt}%
% 		\colorbox{litegrey}{\TheSbox}\end{center}}

\newenvironment{codesnippet}{%
	\begin{Sbox}\begin{minipage}{\textwidth}\sffamily\small}%
	{\end{minipage}\end{Sbox}%
		\begin{center}%
		\colorbox{litegrey}{\TheSbox}\end{center}}



\usepackage{fancyhdr}
\pagestyle{fancy}

%\renewcommand{\chaptermark}[1]{\markboth{#1}{}}
\renewcommand{\sectionmark}[1]{\markright{\thesection\ - #1}}

\fancyhf{}

\fancyhead[LO]{Sección \rightmark} % \thesection\
\fancyfoot[LO]{\small{Confalonieri, Mignanelli, Suárez}}
\fancyfoot[RO]{\thepage}
\renewcommand{\headrulewidth}{0.5pt}
\renewcommand{\footrulewidth}{0.5pt}
\setlength{\hoffset}{-0.8in}
\setlength{\textwidth}{16cm}
%\setlength{\hoffset}{-1.1cm}
%\setlength{\textwidth}{16cm}
\setlength{\headsep}{0.5cm}
\setlength{\textheight}{25cm}
\setlength{\voffset}{-0.7in}
\setlength{\headwidth}{\textwidth}
\setlength{\headheight}{13.1pt}

\renewcommand{\baselinestretch}{1.1}  % line spacing


\usepackage{underscore}
\usepackage{caratula}
\usepackage{url}
\usepackage{color}
\usepackage{clrscode3e} % necesario para el pseudocodigo (estilo Cormen)




\begin{document}
%
%\lstset{
%  language=C++,                    % (cambiar al lenguaje correspondiente)
%  backgroundcolor=\color{white},   % choose the background color
%  basicstyle=\footnotesize,        % size of fonts used for the code
%  breaklines=true,                 % automatic line breaking only at whitespace
%  captionpos=b,                    % sets the caption-position to bottom
%  commentstyle=\color{red},    % comment style
%  escapeinside={\%*}{*)},          % if you want to add LaTeX within your code
%  keywordstyle=\color{blue},       % keyword style
%  stringstyle=\color{blue},     % string literal style
%}

\thispagestyle{empty}
\materia{Teoría de Lenguajes}
\submateria{Primer Cuatrimestre 2016}
\titulo{La clausura a LR(1) en 80 items}
%\subtitulo{Planos de Corte (INSERTE MEJORAS)}
\integrante{Confalonieri, Gisela Belén}{511/11}{gise_5291@yahoo.com.ar} % por cada integrante (apellido, nombre) (n° libreta) (e-mail)
\integrante{Mignanelli, Alejandro Rubén}{609/11}{minga_titere@hotmail.com} 
\integrante{Suárez, Federico}{610/11}{elgeniofederico@gmail.com} 

\maketitle
\newpage

\thispagestyle{empty}
\vfill
%\begin{abstract}
%    \vspace{0.5cm}
%	
%
%\end{abstract}

\thispagestyle{empty}
\vspace{1.5cm}
\tableofcontents
\newpage

%\normalsize
 
\newpage

\section{Introducción}

En el presente trabajo, se pretende desarrollar un analizador léxico y sintáctico para un lenguaje de scripting, denominado Simple Lenguaje de Scripting (SLS). El mismo, recibirá un codigo fuente como entrada, y deberá chequear si cumple la sintaxis y restricciones de tipado del lenguaje, para luego formatear el código con la ‘indentación’ adecuada para SLS. En caso de haberse detectado algún error, se informará claramente cuáles son las características del mismo.
Las características del SLS se encuentran en el enunciado de este trabajo práctico.


%\newpage

\section{La gramatica}

En esta sección se muestra cómo se realizó la tokenización de las expresiones válidas en este leguaje, una primer gramática que acepta estas expresiones, y las transformaciones que se realizaron a la misma hasta llegar a la gramática implementada.

\subsection{Tokens}

En este pequeño lenguaje de scripting, existen palabras reservadas que no pueden usarse como nombres de variables, tanto en mayúsculas como en minúsculas, y el proceso de tokenización tiene en cuenta este aspecto.

El Cuadro \ref{tab-tokens} describe, para cada token definido, el símbolo que representa y la expresión regular asociada, con el formato aceptado por Python.

\begin{table}[!htb]
\begin{center}
\begin{tabular}{| l | l | l |}
\hline
{\bf TOKEN} & {\bf Símbolo representado} & {\bf Expr Regular}\\
\hline
\hline
STR 	& cadena de caracteres entre comillas dobles & "[$\wedge$"]*" \\
\hline
BOOL & true o false & ([tT][rR][uU][eE] $\mid$ \\

	& 				& [fF][aA][lL][sS][eE]) \\
\hline
NUM 	& cualquier cadena numérica & [0-9]+ \\
\hline
VAR 	& cadena alfanumérica con '_' que  & [a-zA-Z][a-zA-Z0-9_] \\
	& comienza en una letra				&	\\
\hline
PUNTO & '.' & . \\
\hline
DOSPTOS	& ':' & : \\
\hline
COMA & ',' & , \\
\hline
ADM & '!' & ! \\
\hline
PREG & '?' & ? \\
\hline
PTOCOMA 	& ';' & ; \\
\hline
LCORCH & '[' & [ \\
\hline
RCORCH & ']' & ] \\
\hline
LPAREN & '(' & ( \\
\hline
RPAREN & ')' & ) \\
\hline
LLLAVE & '\{' & \{ \\
\hline
RLLAVE& '\}' & \} \\
\hline
MAS & '+' & + \\
\hline
MENOS & '-' & - \\
\hline
IGUAL & '=' & = \\
\hline
POR & '*' & * \\
\hline
DIV & '/' & / \\
\hline
POT & '$\wedge$' & $\wedge$  \\
\hline
MOD & '\% ' & \% \\
\hline
MAYOR & '$>$' & $>$ \\
\hline
MENOR & '$<$' & $<$ \\
\hline
COMENT &'\# ' y cualquier cadena de caracteres, & \#.* \\
		&  hasta el primer salto de línea		&	\\
\hline
BEGIN & 'begin' CCMM &  [bB][Ee][gG][iI][nN]\\
\hline
END & 'end' CCMM & [eE][nN][dD] \\
\hline
WHILE & 'while' CCMM & [wW][hH][iI][lL][eE] \\
\hline
FOR & 'for' CCMM & [fF][oO][rR] \\
\hline
IF & 'if' CCMM & [iI][fF] \\
\hline
ELSE & 'else' CCMM & [eE][lL][sS][eE] \\
\hline
DO & 'do' CCMM & [dD][Oo] \\
\hline
RES & 'res' CCMM & [rR][eE][sS] \\
\hline
RETURN & 'return' CCMM & [rR][eE][tt][uU][rR][nN] \\
\hline
AND & 'and' CCMM & [aA][nN][dD] \\
\hline
OR & 'or' CCMM & [oO][rR] \\
\hline
NOT 	& 'not' CCMM & [nN][oO][tT] \\
\hline
PRINT & 'print' CCMM & [pP][rR][iI][nN][tT] \\
\hline
MULTESC & 'multiplicacionescalar' CCMM & [mM][uU][lL][tT][iI][pP][lL][iI][cC][aA] \\
 & & [cC][iI][oO][nN][eE][sS][cC][aA][Ll][aA][rR] \\
\hline
CAP & 'capitalizar' CCMM & [cC][aA][Pp][iI][tT][aA][lL][iI][zZ][aA][rR] \\
\hline
COLIN & 'colineales' CCMM & [cC][oO][lL][iI][nN][eE][aA][lL][eE][sS] \\
\hline
LENGTH & 'length' CCMM & [lL][eE][nN][gG][tT][hH] \\
\hline

\end{tabular}
\end{center}
\caption{Tokens de la garmática}\label{tab-tokens}
\end{table}

%\vspace*{0.5cm}

Nota: CCMM significa 'Cualquier Combinación de mayúsculas y minúsculas'

\newpage

\subsection{Primer Gramatica}


\begin{verbatim}



#----------------------------------------------------#
def p_inicial(expr):
	'start : codigo'
#----------------------------------------------------#

def p_constante_valor(cte):
	'''constante : STR	
				| BOOL
				| numero
				| LPAREN constante RPAREN'''

#----------------------------------------------------#

def p_constante_funcion(f):
	'constante : funcion'

#----------------------------------------------------#
def p_variable(expr):
	'''variable : VAR
				| RES
				| VAR LCORCH z RCORCH
				| LPAREN variable RPAREN
				| VAR PUNTO VAR'''
#----------------------------------------------------#
def p_numero(num):
	'''numero : NUM
			| NUM PUNTO NUM
			| MAS NUM
			| MAS NUM PUNTO NUM
			| MENOS NUM
			| MENOS NUM PUNTO NUM'''
#----------------------------------------------------#
def p_zeta(expr):
	'''z : zso
		| operacion'''
#----------------------------------------------------#

def p_zeta_sin_oper(expr):
	'''zso : variable
			| constante
			| vector
			| registro'''

#----------------------------------------------------#

def p_ge(expr):
	'''g : variable
		| constante 
		| relacion
		| logico'''

#----------------------------------------------------#
def p_vector(expr):
	'''vector : LCORCH z separavec RCORCH
			| LPAREN vector RPAREN'''			
#----------------------------------------------------#

def p_separavector(expr):
	'''separavec : empty
				| COMA z separavec'''

#----------------------------------------------------#
def p_registro(expr):
	'''registro : LLLAVE RLLAVE
				| LLLAVE VAR DOSPTOS z separareg RLLAVE
				| LPAREN registro RPAREN'''

#----------------------------------------------------#

def p_separaregistro(expr):
	'''separareg : empty
				| COMA VAR DOSPTOS z separareg'''

#----------------------------------------------------#

def p_asignacion(expr):
	'''asignacion : variable operasig z
				| variable operasig ternario'''
#----------------------------------------------------#

def p_operasig(op):
	'''operasig : IGUAL
				| MAS IGUAL
				| MENOS IGUAL
				| POR IGUAL
				| DIV IGUAL'''

#----------------------------------------------------#

def p_matematico(expr):
	'''matematico : matprim operMatBinario matf
				| LPAREN matematico RPAREN'''

#----------------------------------------------------#

def p_matprim(expr):		   
	'''matprim : matprim operMatBinario matf
				| matf'''

#----------------------------------------------------#

def p_matf(expr):
	'''matf : zso
			| LPAREN matematico RPAREN'''

#----------------------------------------------------#

def p_operMatBinario(op):
	'''operMatBinario : MAS
					| MENOS
					| POR
					| POT
					| MOD
					| DIV'''

#----------------------------------------------------#

def p_autoincdec(expr):
	'''autoincdec : operMatUnario variable
				| variable operMatUnario'''

#----------------------------------------------------#


def p_operMatUnario(op):
	'''operMatUnario : MAS MAS
					| MENOS MENOS'''

#----------------------------------------------------#

def p_relacion(expr):
	'''relacion : relprim operRelacion relf
			  | LPAREN relacion RPAREN'''

#----------------------------------------------------#

def p_relprim(expr):		   
	'''relprim : relprim operRelacion relf
				| relf'''

#----------------------------------------------------#

def p_relf(expr):
	'''relf : zso
			| matematico
			| LPAREN relacion RPAREN
			| LPAREN logico RPAREN'''

#----------------------------------------------------#

def p_operRelacion(op):
	'''operRelacion : IGUAL IGUAL
					| ADM IGUAL
					| MAYOR
					| MENOR'''


#----------------------------------------------------#

def p_logico(expr):
	'''logico : logprim operLogicoBinario logf
			  | LPAREN logico RPAREN
			  | NOT z'''

#----------------------------------------------------#

def p_logprim(expr):
	'''logprim : logprim operLogicoBinario logf
				| logf'''

#----------------------------------------------------#

def p_logf(expr):
	'''logf : zso
			| relacion
			| LPAREN logico RPAREN'''


#----------------------------------------------------#

def p_operLogBinario(op):
	'''operLogicoBinario : AND
						| OR'''

#----------------------------------------------------#

def p_ternario(expr):
	'''ternario : g PREG z DOSPTOS z
				| g PREG ternario DOSPTOS ternario'''


#----------------------------------------------------#

def p_operacion(expr):
	'''operacion : matematico
				| relacion
				| logico'''


#----------------------------------------------------#

def p_sentencia_(expr):
	'''sentencia : asignacion PTOCOMA
				| PRINT z PTOCOMA
				| autoincdec PTOCOMA'''

#----------------------------------------------------#

def p_funcion_multesc(expr):
	'''funcion : MULTESC LPAREN z COMA z RPAREN
				| MULTESC LPAREN z COMA z COMA z RPAREN'''

#----------------------------------------------------#

def p_funcion_cap(expr):
	'funcion : CAP LPAREN z RPAREN'

#----------------------------------------------------#

def p_funcion_colin(expr):
	'funcion : COLIN LPAREN z COMA z RPAREN'

#----------------------------------------------------#

def p_funcion_length(expr):
	'funcion : LENGTH LPAREN z RPAREN'

#----------------------------------------------------#

def p_stmt(expr):
	'''stmt : closedstmt
			| openstmt'''

#----------------------------------------------------#

def p_closedstmt(expr):
	'''closedstmt : sentencia 
				| LLLAVE codigo RLLAVE
				| dowhile 
				| IF LPAREN g RPAREN closedstmt ELSE closedstmt
				| loopheader closedstmt
				| comentario closedstmt
				'''


#----------------------------------------------------#

def p_openstmt(expr):
	'''openstmt : IF LPAREN g RPAREN stmt
				| IF LPAREN g RPAREN closedstmt ELSE openstmt
				| loopheader openstmt
				| comentario openstmt'''

#----------------------------------------------------#

def p_bucle(expr):
	'''loopheader : for
				| while'''

#----------------------------------------------------#

def p_for_sinasig(expr):
	'''for : FOR LPAREN PTOCOMA g PTOCOMA RPAREN
			| FOR LPAREN PTOCOMA g PTOCOMA asignacion RPAREN 
			| FOR LPAREN PTOCOMA g PTOCOMA autoincdec RPAREN'''

#----------------------------------------------------#

def p_for_conasig(expr):
	'''for : FOR LPAREN asignacion PTOCOMA g PTOCOMA RPAREN
			| FOR LPAREN asignacion PTOCOMA g PTOCOMA asignacion RPAREN
			| FOR LPAREN asignacion PTOCOMA g PTOCOMA autoincdec RPAREN'''


#----------------------------------------------------#

def p_while(expr):
	'while : WHILE LPAREN g RPAREN'

#----------------------------------------------------#

def p_dowhile(expr):
	'dowhile : DO stmt WHILE LPAREN g RPAREN PTOCOMA'

#----------------------------------------------------#

def p_codigo(expr):
	'''codigo : stmt codigo
		    | stmt
		    | comentario'''

#----------------------------------------------------#

def p_comentario(expr):
	'comentario : COMENT'

#----------------------------------------------------#

def p_empty(p):
	'empty :'
	pass

#----------------------------------------------------#

def p_error(token):
    message = "[Syntax error]"
    if token is not None:
        message += "\ntype:" + token.type
        message += "\nvalue:" + str(token.value)
        message += "\nline:" + str(token.lineno)
        message += "\nposition:" + str(token.lexpos)
    raise Exception(message)


#----------------------------------------------------#

\end{verbatim}


\newpage

\section{La Solución}

En el Apéndice 1 se encuentra el código fuente del parser.

\subsection{Atributos}

Los no terminales de la gramática fueron asociados a clases creadas en Python, cada una con ciertos atributos necesarios para el análisis de tipado y el formateo del código.

\begin{itemize}
\item Las variables se asocian a la clase Variable, que contiene los siguientes atributos:
	\begin{itemize}
	\item {\it nombre} Contiene un string con el nombre de la variable.
	\item {\it impr} Contiene un string con el formato imprimible de la variable.
	\item {\it campo} Contiene un string con el nombre del campo, en caso de que se trate de una variable del tipo {\tt registro.campo}.
	\item {\it array_elem} Contiene un entero (mayor o igual a 0) que indica el nivel de anidamiento de índices en variables del tipo {\tt variable[indice]}. Por ejemplo, la variable {\tt a} tendrá este atributo en 0, la variable b[0] tendrá este atributo en 1 y la variable c[2][4][6] tendrá este atributo en 3.
	\end{itemize}
\item Los números, las funciones, operaciones y vectores se asocian a subclases de la clase Constante, que contiene los siguientes atributos:
	\begin{itemize}
	\item {\it tipo} Contiene un string con el tipo del elemento.
	\item {\it impr} Contiene un string con el formato imprimible del elemento.
	\end{itemize}
\item Los registros se asocian a la clase Registro, que contiene los siguientes atributos:
	\begin{itemize}
	\item {\it tipo} Es constantemente 'registro'.
	\item {\it impr} Contiene un string con el formato imprimible del registro.
	\end{itemize}
\item Las sentencias, bloques de código, ciclos, condicionales y comentarios se asocian a la clase Código, que contiene los siguientes atributos:
	\begin{itemize}
	\item {\it llaves} Contiene un entero (0 o 1) que indica si se trata de un bloque encerrado entre llaves o no.
	\item {\it impr} Contiene un string con el formato imprimible del registro.
	\end{itemize}
\end{itemize}


\subsection{Manejo de variables}

Para el correcto chequeo de tipos de las variables, como el tipo de las mismas puede variar, se decidió implementar un diccionario global \{variable : tipo\}.  Las claves del diccinario son los nombres de variable, y los valores son strings con el tipo correspondiente.

Un caso especial son los registros.  Para ello, aprovechando la versatilidad en cuanto a tipado de Python, el valor asociado a una variable de tipo registro es un nuevo diccionario \{campo : tipo\}, cuyas claves son los campos del registro, y sus valores son los tipos correspondientes.

Por otro lado, si se trata de una variable de tipo vector, el tipo que se coloca en el diccionario es {\tt vectortipo}, donde {\tt tipo} es el tipo de los elementos del vector.  Esto es importante al chequear, por ejemplo, que si se declara un vector de enteros, no se pueda asignar a una posición del mismo una cadena.

Al cambiar el tipo de una variable, o del campo de un registro, simplemente se reemplaza el valor en el diccionario.
%
%\subsection{Manejo de palabras reservadas}
%
%Como se mencionó anteriormente, el tokenizador evita que las variables tomen nombres de palabras reservadas, escritas con cualquier combinación de mayúsculas y minúsculas.  Estas palabras reservadas son enviadas a su token correspondiente.
%
%Por cuestiones prácticas, se tomó la decisión de 'normalizar' las palabras reservadas al dar el formato final al código.  Esta 'normalización' lleva las palabras reservadas a minúsculas, independientemente de cómo fueron escritas en el código original, salvo la función {\tt multiplicacionEscalar} que contiene la 'E' mayúscula, y los operadores lógicos {\tt AND, OR} Y {\tt NOT} que se llevan a mayúsculas.
%
%En cuanto a la variable de sistema {\tt res}, también se normaliza a minúsculas, y si se utiliza repetidas veces en el código original, variando la combinación de mayúsculas y minúsculas, se asocia siempre a la misma variable.  Por ejemplo, {\tt res = 0;} y {\tt ReS = 1;} refieren a la misma variable.  Esto es importante a la hora de chequear los tipos.

\subsection{Problemas y decisiones}

\subsubsection{Vectores de registros}

En el caso de declararse un vector de registros, no existe al momento un control que permita consultar un campo de un registro de dicho vector. Por ejemplo, el siguiente código falla:

\begin{lstlisting}
usuario1 = {nombre:"Al", edad:50};
usuario2 = {nombre:"Mr.X", edad:10};
usuarios = [usuario1, usuario2];

s = usuarios[1].edad;
\end{lstlisting}

\subsubsection{Decisiones}
Se han tomado las siguientes decisiones respecto al código que debe aceptarse:

\begin{itemize}
	\item {\bf Indexar vectores: } Para acceder a una posición de un vector, el mismo debe haber sido asignado previamente a una variable, para luego pedir el índice correspondiente.  Por lo tanto, un código como el siguiente falla:
\begin{lstlisting}
hola = [1,2,3,4,5,6][4];
\end{lstlisting}
	\item {\bf Utilización de registros: } Los registros definidos como campos entre llaves deben asignarse a una variable.  No se acepta utilizar un registro o acceder a sus campos si no es a través de una variable.
	\item {\bf Operador ternario: } Este operador sólo tiene sentido si se lo asigna directamente a una variable.  Por ejemplo, el siguiente código falla:
\begin{lstlisting}
a = 3 + (true ? 3 : 4);
\end{lstlisting}
	\item {\bf Funciones: } Excepto la función {\tt print}, ninguna otra función puede utilizarse como sentencia, es decir, sin usar su resultado en una asignación o una operación, dado que ninguna función modifica los datos de entrada.  Por lo tanto, un código como el que sigue falla:
\begin{lstlisting}
length("pepe");
\end{lstlisting}
	\item {\bf Números negativos: } Para que un número sea considerado un negativo, debe colocarse el signo (-) inmediatamente antes del número en cuestión.  Si el número está entre paréntesis, el signo será considerado como operación de resta.  De la misma manera, una variable de tipo numérico será negativa sólo si se le asigna un valor negativo. Por ejemplo, el siguiente código falla:
\begin{lstlisting}
a = -(7);
x = 3;
b = -x;
\end{lstlisting}
	\item {\bf Autoincremento y autodecremento: } Los operadores (++) y (--) sólo aplica a variables, y de tipo entero.  Es decir, no se pueden utilizar sobre números o sobre variables de tipo float.
\end{itemize} 

\newpage

\section{Ejecución}

\subsection{Requerimientos}

Para ejecutar el parser se requiere contar con el siguiente software:

\begin{itemize}
\item Python 2.7 (o superior)
\item Python-PLY 3.6 (o superior)
\end{itemize}

Las pruebas se realizaron sobre Ubuntu 14.04.

\subsection{Cómo correr}

Se provee el código fuente de este parser, el cual consta de los siguientes archivos:

\begin{itemize}
\item expressions.py
\item lexer_rules.py
\item parser_rules.py
\item parser.py
\item SLSParser.sh
\end{itemize}

Para correr el programa, es necesario que todos los archivos fuente se encuentren en el mismo directorio.



\newpage

\section{Casos de Prueba}


\newpage

\section{Resultados}


\newpage

\section{Conclusiones}

Consideramos este trabajo un interesante cierre para la materia, puesto que nos permitió trabajar con una gramática amplia, que presenta mayores desafíos que solucionar un concepto en una gramática con 3 no terminales.

A nivel sintáctico, se presentó un gran desafío para evitar que la gramática sea ambigua, como por ejemplo el problema del dangling else, que no fue solucionado solamente con la respuesta de un libro, ya que la amplitud de la gramática causó conflictos en otras regiones, y requirió astucia y perseverancia hasta acomodar la gramática de manera tal que esta no tenga ambigüedades.

A nivel semántico, se tuvieron que tomar varias decisiones que debían congeniar y formar una estructura coherente, lo cual en una gramática de esta amplitud no fue trivial.

En conclusión, fue un trabajo muy interesante y con más desafíos de los esperados.

\newpage

\section{Apendice 1: Código}

\lstset{language=Python, breaklines=true, basicstyle=\scriptsize, commentstyle=\color{blue}, numbers=left, stepnumber=1, tabsize=4}
\lstinputlisting[language=Python]{../parserrules.py}

\end{document}

