\section{La gramatica}

En esta sección se muestra cómo se realizó la tokenización de las expresiones válidas en este leguaje, una primer gramática que acepta estas expresiones, y las transformaciones que se realizaron a la misma hasta llegar a la gramática implementada.

\subsection{Tokens}

En este pequeño lenguaje de scripting, existen palabras reservadas que no pueden usarse como nombres de variables, tanto en mayúsculas como en minúsculas, y el proceso de tokenización tiene en cuenta este aspecto.

El Cuadro \ref{tab-tokens} describe, para cada token definido, el símbolo que representa y la expresión regular asociada, con el formato aceptado por Python.

\begin{table}[!htb]
\begin{center}
\begin{tabular}{| l | l | l |}
\hline
{\bf TOKEN} & {\bf Símbolo representado} & {\bf Expr Regular}\\
\hline
\hline
STR 	& cadena de caracteres entre comillas dobles & "[$\wedge$"]*" \\
\hline
BOOL & true o false & ([tT][rR][uU][eE] $\mid$ \\

	& 				& [fF][aA][lL][sS][eE]) \\
\hline
NUM 	& cualquier cadena numérica & [0-9]+ \\
\hline
VAR 	& cadena alfanumérica con '_' que  & [a-zA-Z][a-zA-Z0-9_] \\
	& comienza en una letra				&	\\
\hline
PUNTO & '.' & . \\
\hline
DOSPTOS	& ':' & : \\
\hline
COMA & ',' & , \\
\hline
ADM & '!' & ! \\
\hline
PREG & '?' & ? \\
\hline
PTOCOMA 	& ';' & ; \\
\hline
LCORCH & '[' & [ \\
\hline
RCORCH & ']' & ] \\
\hline
LPAREN & '(' & ( \\
\hline
RPAREN & ')' & ) \\
\hline
LLLAVE & '\{' & \{ \\
\hline
RLLAVE& '\}' & \} \\
\hline
MAS & '+' & + \\
\hline
MENOS & '-' & - \\
\hline
IGUAL & '=' & = \\
\hline
POR & '*' & * \\
\hline
DIV & '/' & / \\
\hline
POT & '$\wedge$' & $\wedge$  \\
\hline
MOD & '\% ' & \% \\
\hline
MAYOR & '$>$' & $>$ \\
\hline
MENOR & '$<$' & $<$ \\
\hline
COMENT &'\# ' y cualquier cadena de caracteres, & \#.* \\
		&  hasta el primer salto de línea		&	\\
\hline
BEGIN & 'begin' CCMM &  [bB][Ee][gG][iI][nN]\\
\hline
END & 'end' CCMM & [eE][nN][dD] \\
\hline
WHILE & 'while' CCMM & [wW][hH][iI][lL][eE] \\
\hline
FOR & 'for' CCMM & [fF][oO][rR] \\
\hline
IF & 'if' CCMM & [iI][fF] \\
\hline
ELSE & 'else' CCMM & [eE][lL][sS][eE] \\
\hline
DO & 'do' CCMM & [dD][Oo] \\
\hline
RES & 'res' CCMM & [rR][eE][sS] \\
\hline
RETURN & 'return' CCMM & [rR][eE][tt][uU][rR][nN] \\
\hline
AND & 'and' CCMM & [aA][nN][dD] \\
\hline
OR & 'or' CCMM & [oO][rR] \\
\hline
NOT 	& 'not' CCMM & [nN][oO][tT] \\
\hline
PRINT & 'print' CCMM & [pP][rR][iI][nN][tT] \\
\hline
MULTESC & 'multiplicacionescalar' CCMM & [mM][uU][lL][tT][iI][pP][lL][iI][cC][aA] \\
 & & [cC][iI][oO][nN][eE][sS][cC][aA][Ll][aA][rR] \\
\hline
CAP & 'capitalizar' CCMM & [cC][aA][Pp][iI][tT][aA][lL][iI][zZ][aA][rR] \\
\hline
COLIN & 'colineales' CCMM & [cC][oO][lL][iI][nN][eE][aA][lL][eE][sS] \\
\hline
LENGTH & 'length' CCMM & [lL][eE][nN][gG][tT][hH] \\
\hline

\end{tabular}
\end{center}
\caption{Tokens de la garmática}\label{tab-tokens}
\end{table}

%\vspace*{0.5cm}

Nota: CCMM significa 'Cualquier Combinación de mayúsculas y minúsculas'

\newpage

\subsection{Primer Gramatica}


\begin{verbatim}



#----------------------------------------------------#
def p_inicial(expr):
	'start : codigo'
#----------------------------------------------------#

def p_constante_valor(cte):
	'''constante : STR	
				| BOOL
				| numero
				| LPAREN constante RPAREN'''

#----------------------------------------------------#

def p_constante_funcion(f):
	'constante : funcion'

#----------------------------------------------------#
def p_variable(expr):
	'''variable : VAR
				| RES
				| VAR LCORCH z RCORCH
				| LPAREN variable RPAREN
				| VAR PUNTO VAR'''
#----------------------------------------------------#
def p_numero(num):
	'''numero : NUM
			| NUM PUNTO NUM
			| MAS NUM
			| MAS NUM PUNTO NUM
			| MENOS NUM
			| MENOS NUM PUNTO NUM'''
#----------------------------------------------------#
def p_zeta(expr):
	'''z : zso
		| operacion'''
#----------------------------------------------------#

def p_zeta_sin_oper(expr):
	'''zso : variable
			| constante
			| vector
			| registro'''

#----------------------------------------------------#

def p_ge(expr):
	'''g : variable
		| constante 
		| relacion
		| logico'''

#----------------------------------------------------#
def p_vector(expr):
	'''vector : LCORCH z separavec RCORCH
			| LPAREN vector RPAREN'''			
#----------------------------------------------------#

def p_separavector(expr):
	'''separavec : empty
				| COMA z separavec'''

#----------------------------------------------------#
def p_registro(expr):
	'''registro : LLLAVE RLLAVE
				| LLLAVE VAR DOSPTOS z separareg RLLAVE
				| LPAREN registro RPAREN'''

#----------------------------------------------------#

def p_separaregistro(expr):
	'''separareg : empty
				| COMA VAR DOSPTOS z separareg'''

#----------------------------------------------------#

def p_asignacion(expr):
	'''asignacion : variable operasig z
				| variable operasig ternario'''
#----------------------------------------------------#

def p_operasig(op):
	'''operasig : IGUAL
				| MAS IGUAL
				| MENOS IGUAL
				| POR IGUAL
				| DIV IGUAL'''

#----------------------------------------------------#

def p_matematico(expr):
	'''matematico : matprim operMatBinario matf
				| LPAREN matematico RPAREN'''

#----------------------------------------------------#

def p_matprim(expr):		   
	'''matprim : matprim operMatBinario matf
				| matf'''

#----------------------------------------------------#

def p_matf(expr):
	'''matf : zso
			| LPAREN matematico RPAREN'''

#----------------------------------------------------#

def p_operMatBinario(op):
	'''operMatBinario : MAS
					| MENOS
					| POR
					| POT
					| MOD
					| DIV'''

#----------------------------------------------------#

def p_autoincdec(expr):
	'''autoincdec : operMatUnario variable
				| variable operMatUnario'''

#----------------------------------------------------#


def p_operMatUnario(op):
	'''operMatUnario : MAS MAS
					| MENOS MENOS'''

#----------------------------------------------------#

def p_relacion(expr):
	'''relacion : relprim operRelacion relf
			  | LPAREN relacion RPAREN'''

#----------------------------------------------------#

def p_relprim(expr):		   
	'''relprim : relprim operRelacion relf
				| relf'''

#----------------------------------------------------#

def p_relf(expr):
	'''relf : zso
			| matematico
			| LPAREN relacion RPAREN
			| LPAREN logico RPAREN'''

#----------------------------------------------------#

def p_operRelacion(op):
	'''operRelacion : IGUAL IGUAL
					| ADM IGUAL
					| MAYOR
					| MENOR'''


#----------------------------------------------------#

def p_logico(expr):
	'''logico : logprim operLogicoBinario logf
			  | LPAREN logico RPAREN
			  | NOT z'''

#----------------------------------------------------#

def p_logprim(expr):
	'''logprim : logprim operLogicoBinario logf
				| logf'''

#----------------------------------------------------#

def p_logf(expr):
	'''logf : zso
			| relacion
			| LPAREN logico RPAREN'''


#----------------------------------------------------#

def p_operLogBinario(op):
	'''operLogicoBinario : AND
						| OR'''

#----------------------------------------------------#

def p_ternario(expr):
	'''ternario : g PREG z DOSPTOS z
				| g PREG ternario DOSPTOS ternario'''


#----------------------------------------------------#

def p_operacion(expr):
	'''operacion : matematico
				| relacion
				| logico'''


#----------------------------------------------------#

def p_sentencia_(expr):
	'''sentencia : asignacion PTOCOMA
				| PRINT z PTOCOMA
				| autoincdec PTOCOMA'''

#----------------------------------------------------#

def p_funcion_multesc(expr):
	'''funcion : MULTESC LPAREN z COMA z RPAREN
				| MULTESC LPAREN z COMA z COMA z RPAREN'''

#----------------------------------------------------#

def p_funcion_cap(expr):
	'funcion : CAP LPAREN z RPAREN'

#----------------------------------------------------#

def p_funcion_colin(expr):
	'funcion : COLIN LPAREN z COMA z RPAREN'

#----------------------------------------------------#

def p_funcion_length(expr):
	'funcion : LENGTH LPAREN z RPAREN'

#----------------------------------------------------#

def p_stmt(expr):
	'''stmt : closedstmt
			| openstmt'''

#----------------------------------------------------#

def p_closedstmt(expr):
	'''closedstmt : sentencia 
				| LLLAVE codigo RLLAVE
				| dowhile 
				| IF LPAREN g RPAREN closedstmt ELSE closedstmt
				| loopheader closedstmt
				| comentario closedstmt
				'''


#----------------------------------------------------#

def p_openstmt(expr):
	'''openstmt : IF LPAREN g RPAREN stmt
				| IF LPAREN g RPAREN closedstmt ELSE openstmt
				| loopheader openstmt
				| comentario openstmt'''

#----------------------------------------------------#

def p_bucle(expr):
	'''loopheader : for
				| while'''

#----------------------------------------------------#

def p_for_sinasig(expr):
	'''for : FOR LPAREN PTOCOMA g PTOCOMA RPAREN
			| FOR LPAREN PTOCOMA g PTOCOMA asignacion RPAREN 
			| FOR LPAREN PTOCOMA g PTOCOMA autoincdec RPAREN'''

#----------------------------------------------------#

def p_for_conasig(expr):
	'''for : FOR LPAREN asignacion PTOCOMA g PTOCOMA RPAREN
			| FOR LPAREN asignacion PTOCOMA g PTOCOMA asignacion RPAREN
			| FOR LPAREN asignacion PTOCOMA g PTOCOMA autoincdec RPAREN'''


#----------------------------------------------------#

def p_while(expr):
	'while : WHILE LPAREN g RPAREN'

#----------------------------------------------------#

def p_dowhile(expr):
	'dowhile : DO stmt WHILE LPAREN g RPAREN PTOCOMA'

#----------------------------------------------------#

def p_codigo(expr):
	'''codigo : stmt codigo
		    | stmt
		    | comentario'''

#----------------------------------------------------#

def p_comentario(expr):
	'comentario : COMENT'

#----------------------------------------------------#

def p_empty(p):
	'empty :'
	pass

#----------------------------------------------------#

def p_error(token):
    message = "[Syntax error]"
    if token is not None:
        message += "\ntype:" + token.type
        message += "\nvalue:" + str(token.value)
        message += "\nline:" + str(token.lineno)
        message += "\nposition:" + str(token.lexpos)
    raise Exception(message)


#----------------------------------------------------#

\end{verbatim}