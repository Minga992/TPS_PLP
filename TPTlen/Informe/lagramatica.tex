\section{La gramatica}

En esta sección se muestra cómo se realizó la tokenización de las expresiones válidas en este leguaje, la gramática implementada y algunas aclaraciones sobre los conflictos con los que nos encontramos en la medida que la creábamos.

\subsection{Tokens}

En este pequeño lenguaje de scripting, existen palabras reservadas que no pueden usarse como nombres de variables, en cualquier combinación de minúsculas y mayúsculas: {\tt begin, end, while, for,
if, else, do, res, return, true, false, AND, OR, NOT}. El proceso de tokenización tiene en cuenta este aspecto, y en caso de encontrar una palabra reservada, se asigna al token correspondiente o se arroja un error en caso de no existir un token asociado.

El Cuadro \ref{tab-tokens} describe, para cada token definido, el símbolo que representa y la expresión regular asociada, con el formato aceptado por Python.

\begin{table}[!htb]
\begin{center}
\begin{tabular}{| l | l | l |}
\hline
{\bf TOKEN} & {\bf Símbolo representado} & {\bf Expr Regular}\\
\hline
\hline
STR 	& cadena de caracteres entre comillas dobles & "[$\wedge$"]*" \\
\hline
BOOL & true o false & (true $\mid$ false) \\
\hline
NUM 	& cualquier cadena numérica & [0-9]+ \\
\hline
VAR 	& cadena alfanumérica con '_' que comienza en una letra & [a-zA-Z][a-zA-Z0-9_] \\
\hline
PUNTO & '.' & . \\
\hline
DOSPTOS	& ':' & : \\
\hline
COMA & ',' & , \\
\hline
ADM & '!' & ! \\
\hline
PREG & '?' & ? \\
\hline
PTOCOMA 	& ';' & ; \\
\hline
LCORCH & '[' & [ \\
\hline
RCORCH & ']' & ] \\
\hline
LPAREN & '(' & ( \\
\hline
RPAREN & ')' & ) \\
\hline
LLLAVE & '\{' & \{ \\
\hline
RLLAVE& '\}' & \} \\
\hline
MAS & '+' & + \\
\hline
MENOS & '-' & - \\
\hline
IGUAL & '=' & = \\
\hline
POR & '*' & * \\
\hline
DIV & '/' & / \\
\hline
POT & '$\wedge$' & $\wedge$  \\
\hline
MOD & '\% ' & \% \\
\hline
MAYOR & '$>$' & $>$ \\
\hline
MENOR & '$<$' & $<$ \\
\hline
COMENT &'\# ' y cualquier cadena de caracteres, hasta el primer salto de línea & \#.* \\
\hline
WHILE & 'while' & while \\
\hline
FOR & 'for' & for \\
\hline
IF & 'if' & if \\
\hline
ELSE & 'else' & else \\
\hline
DO & 'do' & do \\
\hline
RES & 'res' & res \\
\hline
AND & 'AND'& AND \\
\hline
OR & 'OR' & OR \\
\hline
NOT 	& 'NOT' & NOT \\
\hline
PRINT & 'print' & print \\
\hline
MULTESC & 'multiplicacionEscalar' & multiplicacionEscalar\\
\hline
CAP & 'capitalizar' & capitalizar \\
\hline
COLIN & 'colineales' & colineales \\
\hline
LENGTH & 'length' & length \\
\hline

\end{tabular}
\end{center}
\caption{Tokens de la garmática}\label{tab-tokens}
\end{table}

%\vspace*{0.5cm}

%\newpage

\subsection{Gramática}

A continuación se muestran las reglas de la gramática implementada.  La misma es por lo menos LALR, dado la herramienta PLY pudo generar esta tabla sin conflictos a partir de las reglas declaradas.

\subsubsection{Reglas}

%\begin{verbatim}
%#----------------------------------------------------#
%def p_inicial(expr):
%	'start : codigo'
%#----------------------------------------------------#
%def p_constante_valor(cte):
%	'''constante : STR	
%				| BOOL
%				| numero
%				| LPAREN constante RPAREN'''
%#----------------------------------------------------#
%def p_constante_funcion(f):
%	'constante : funcion'
%#----------------------------------------------------#
%def p_variable(expr):
%	'''variable : VAR
%				| RES
%				| VAR LCORCH z RCORCH
%				| LPAREN variable RPAREN
%				| VAR PUNTO VAR'''
%#----------------------------------------------------#
%def p_numero(num):
%	'''numero : NUM
%			| NUM PUNTO NUM
%			| MAS NUM
%			| MAS NUM PUNTO NUM
%			| MENOS NUM
%			| MENOS NUM PUNTO NUM'''
%#----------------------------------------------------#
%def p_zeta(expr):
%	'''z : zso
%		| operacion'''
%#----------------------------------------------------#
%def p_zeta_sin_oper(expr):
%	'''zso : variable
%			| constante
%			| vector
%			| registro'''
%#----------------------------------------------------#
%def p_ge(expr):
%	'''g : variable
%		| constante 
%		| relacion
%		| logico'''
%#----------------------------------------------------#
%def p_vector(expr):
%	'''vector : LCORCH z separavec RCORCH
%			| LPAREN vector RPAREN'''			
%#----------------------------------------------------#
%def p_separavector(expr):
%	'''separavec : empty
%				| COMA z separavec'''
%#----------------------------------------------------#
%def p_registro(expr):
%	'''registro : LLLAVE RLLAVE
%				| LLLAVE VAR DOSPTOS z separareg RLLAVE
%				| LPAREN registro RPAREN'''
%#----------------------------------------------------#
%def p_separaregistro(expr):
%	'''separareg : empty
%				| COMA VAR DOSPTOS z separareg'''
%#----------------------------------------------------#
%def p_asignacion(expr):
%	'''asignacion : variable operasig z
%				| variable operasig ternario'''
%#----------------------------------------------------#
%def p_operasig(op):
%	'''operasig : IGUAL
%				| MAS IGUAL
%				| MENOS IGUAL
%				| POR IGUAL
%				| DIV IGUAL'''
%#----------------------------------------------------#
%def p_matematico(expr):
%	'''matematico : matprim operMatBinario matf
%				| LPAREN matematico RPAREN'''
%#----------------------------------------------------#
%def p_matprim(expr):		   
%	'''matprim : matprim operMatBinario matf
%				| matf'''
%#----------------------------------------------------#
%def p_matf(expr):
%	'''matf : zso
%			| LPAREN matematico RPAREN'''
%#----------------------------------------------------#
%def p_operMatBinario(op):
%	'''operMatBinario : MAS
%					| MENOS
%					| POR
%					| POT
%					| MOD
%					| DIV'''
%#----------------------------------------------------#
%def p_autoincdec(expr):
%	'''autoincdec : operMatUnario variable
%				| variable operMatUnario'''
%#----------------------------------------------------#
%def p_operMatUnario(op):
%	'''operMatUnario : MAS MAS
%					| MENOS MENOS'''
%#----------------------------------------------------#
%def p_relacion(expr):
%	'''relacion : relprim operRelacion relf
%			  | LPAREN relacion RPAREN'''
%#----------------------------------------------------#
%def p_relprim(expr):		   
%	'''relprim : relprim operRelacion relf
%				| relf'''
%#----------------------------------------------------#
%def p_relf(expr):
%	'''relf : zso
%			| matematico
%			| LPAREN relacion RPAREN
%			| LPAREN logico RPAREN'''
%#----------------------------------------------------#
%def p_operRelacion(op):
%	'''operRelacion : IGUAL IGUAL
%					| ADM IGUAL
%					| MAYOR
%					| MENOR'''
%#----------------------------------------------------#
%def p_logico(expr):
%	'''logico : logprim operLogicoBinario logf
%			  | LPAREN logico RPAREN
%			  | NOT z'''
%#----------------------------------------------------#
%def p_logprim(expr):
%	'''logprim : logprim operLogicoBinario logf
%				| logf'''
%#----------------------------------------------------#
%def p_logf(expr):
%	'''logf : zso
%			| relacion
%			| LPAREN logico RPAREN'''
%#----------------------------------------------------#
%def p_operLogBinario(op):
%	'''operLogicoBinario : AND
%						| OR'''
%#----------------------------------------------------#
%def p_ternario(expr):
%	'''ternario : g PREG z DOSPTOS z
%				| g PREG ternario DOSPTOS ternario'''
%#----------------------------------------------------#
%def p_operacion(expr):
%	'''operacion : matematico
%				| relacion
%				| logico'''
%#----------------------------------------------------#
%def p_sentencia_(expr):
%	'''sentencia : asignacion PTOCOMA
%				| PRINT z PTOCOMA
%				| autoincdec PTOCOMA'''
%#----------------------------------------------------#
%def p_funcion_multesc(expr):
%	'''funcion : MULTESC LPAREN z COMA z RPAREN
%				| MULTESC LPAREN z COMA z COMA z RPAREN'''
%#----------------------------------------------------#
%def p_funcion_cap(expr):
%	'funcion : CAP LPAREN z RPAREN'
%#----------------------------------------------------#
%def p_funcion_colin(expr):
%	'funcion : COLIN LPAREN z COMA z RPAREN'
%#----------------------------------------------------#
%def p_funcion_length(expr):
%	'funcion : LENGTH LPAREN z RPAREN'
%#----------------------------------------------------#
%def p_stmt(expr):
%	'''stmt : closedstmt
%			| openstmt'''
%#----------------------------------------------------#
%def p_closedstmt(expr):
%	'''closedstmt : sentencia 
%				| LLLAVE codigo RLLAVE
%				| dowhile 
%				| IF LPAREN g RPAREN closedstmt ELSE closedstmt
%				| loopheader closedstmt
%				| comentario closedstmt
%				'''
%#----------------------------------------------------#
%def p_openstmt(expr):
%	'''openstmt : IF LPAREN g RPAREN stmt
%				| IF LPAREN g RPAREN closedstmt ELSE openstmt
%				| loopheader openstmt
%				| comentario openstmt'''
%#----------------------------------------------------#
%def p_bucle(expr):
%	'''loopheader : for
%				| while'''
%#----------------------------------------------------#
%def p_for_sinasig(expr):
%	'''for : FOR LPAREN PTOCOMA g PTOCOMA RPAREN
%			| FOR LPAREN PTOCOMA g PTOCOMA asignacion RPAREN 
%			| FOR LPAREN PTOCOMA g PTOCOMA autoincdec RPAREN'''
%#----------------------------------------------------#
%def p_for_conasig(expr):
%	'''for : FOR LPAREN asignacion PTOCOMA g PTOCOMA RPAREN
%			| FOR LPAREN asignacion PTOCOMA g PTOCOMA asignacion RPAREN
%			| FOR LPAREN asignacion PTOCOMA g PTOCOMA autoincdec RPAREN'''
%#----------------------------------------------------#
%def p_while(expr):
%	'while : WHILE LPAREN g RPAREN'
%#----------------------------------------------------#
%def p_dowhile(expr):
%	'dowhile : DO stmt WHILE LPAREN g RPAREN PTOCOMA'
%#----------------------------------------------------#
%def p_codigo(expr):
%	'''codigo : stmt codigo
%		    | stmt
%		    | comentario'''
%#----------------------------------------------------#
%def p_comentario(expr):
%	'comentario : COMENT'
%#----------------------------------------------------#
%def p_empty(p):
%	'empty :'
%	pass
%#----------------------------------------------------#
%\end{verbatim}

\lstset{language=Python, breaklines=true, basicstyle=\scriptsize, commentstyle=\color{blue}, tabsize=4}
\lstinputlisting[language=Python]{../parserruleslimpio.py}


\subsubsection{Problemas}

A continuación presentaremos los mayores problemas con los que nos encontramos al generar la gramática apropiada.

\paragraph{Operaciones}
Inicialmente, las reglas para las operaciones binarias (matemáticas, lógicas y relacionales) tenían la siguiente forma:

{\tt operacion : z operador z} (el z de la gramática explicitada)

Claramente, esto generaba muchas ambigüedades, puesto que, por ejemplo $3+3+3$ tenía más de una forma de producirse.

Para resolver este problema se consideró una asociatividad a izquierda para las operaciones del mismo tipo, y la precedencia de las operaciones matemáticas sobre las relacionales, y de éstas sobre las lógicas.  No consideramos la precedencia entre los operadores matemáticos entre sí, lo mismo que entre los lógicos y entre los relacionales.

\paragraph{Dangling Else}
Dado este clásico problema, se recurrió a implementar la solución brindada en 'el libro del Dragón', pero con algunas modificaciones.

Al implementarlo directamente, obtuvimos conflictos con los ciclos 'for' y 'while'.  Por ejemplo:

\begin{lstlisting}
if guarda1
	while guarda2
		if guarda3
			bloque1
		else
			bloque2	
\end{lstlisting}

El problema surgía cuando se daba como input un código donde los bloques poseían una sola sentencia y podían prescindir de las llaves.  En el ejemplo descripto, dada la primera implementación, no podía saberse si el 'else' pertenecía al condicional más grande o al que se encuentra dentro del ciclo.

Esto se producía porque, en una primera aproximación a la gramática, las reglas de estos ciclos tenían la siguiente forma:

{\tt for : FOR (declaracion) bloque} y {\tt while : WHILE (guarda) bloque}

Para solucionar el problema, se decidió quitar el 'bloque' de ambas reglas y manejarlo desde las reglas que definen los bloques de código. 

\paragraph{Comentarios dentro de un condicional}
La gramática actual no acepta una entrada que cuente con un comentario dentro de un condicional, antes de la palabra {\tt else}.  Por ejemplo: 

\begin{lstlisting}
if guarda
	codigo
	#comentario
else
	codigo
\end{lstlisting}