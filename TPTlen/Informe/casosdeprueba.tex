\section{Casos de Prueba}

En esta sección se mostrará un subconjunto de las pruebas que se realizaron para probar el correcto funcionamiento del parser. En caso de que la expresión sintáctica sea correcta, se mostrará lo que el parser develve, y en caso de no serlo se explicará el por que.

\subsection{Casos Sintácticamente Correctos}

\begin{lstlisting}
Prueba1:
a = true;#algo \n b=5; #algo \nfor (;a;b++) #algo \n 
if( a ) #algo \n {b=a; #algo \n if (b) b=a;#algo \n} else c = 8;#algo \n

Resultado:
a = True;
#algo 
b = 5;
#algo 
for( ; a; b++)
	#algo 
	if(a)
		#algo 
		{
		b = a;
		#algo 
			if(b)
				b = a;
		#algo }
	else
		c = 8;
#algo 
\end{lstlisting}

\begin{verbatim}


Prueba2:

a=3; a++; a--; a += 6; a -= 5; a *= 3; print (length( \"pelado\"));
b= (a==8); if (b) a= a; c = a==8? a:a;

Resultado:

a = 3;
a++;
a--;
a += 6;
a -= 5;
a *= 3;
print (length("pelado"));
b = (a == 8);
if(b)
     a = a;
c = a == 8 ? a : a;

Prueba3:
a[3+5] = 2*3; a[length(a)] = 2; a[2] = a [10%3]; b = [3, length(a), a[5]]; b[4] = length(b);

Resultado:

a[3 + 5] = 2 * 3;
a[length(a)] = 2;
a[2] = a[10 % 3];
b = [3, length(a), a[5]];
b[4] = length(b);

\end{verbatim}



\subsection{Casos Sintácticamente Incorrectos}

\begin{verbatim}
Prueba1:

a = [2,3,4]; b = [a]; c = length(b); b[2] = [\"pepe\"];

\end{verbatim}

La razón por la cual esta expresión es sintácticamente incorrecta es que $b$ es un vector de vectores de enteros, por lo cual no puede recibir en ningún subindice un elemento de tipo cadena. Notar que si $b$ no estuviera inicializado, esa asignación se podría hacer, puesto que estaríamos inicializando $b$ en un vector de cadenas.

\begin{verbatim}

Prueba2:

a = {campo : 3}; a.campo = \"pepe\"; b[a.campo] = 2;

\end{verbatim}

En este caso, si bien en un comienzo $a.campo$ era un entero, la segunda asignación cambia su tipo a cadena, por lo cual no puede ser indice del arreglo $b$.

\begin{verbatim}

Prueba3:

campo = [{campo:\"1\"}, {campo:\"2\"}, {campo:\"3\"}][1];

\end{verbatim}

Para referenciar una posición de un vector, se debe acceder a el a través de una variable, por lo cual esta expresión será considerada sintacticamente incorrecta.

